\chapter{Root pairing}
\label{cha:root-pairing}
\section{Basic definitions}
\label{sec:basic-definitions}

We follow the outline of mathlib's definitions. 

\begin{definition}
   \label{def:perfect-pairing}
   \leanok
    A {\it perfect pairing} is a quadruple $(R,M,N,\mathcal{L})$ where $R$ is a commutative ring, $M,N$ are 
    $R$-modules, and $\mathcal{L} : M \times N \to R$ is an $R$-bilinear map such that the induced maps
    \[
        \mathcal{L}_M : M \to {\rm Hom}_R(N,R), \quad m \mapsto \mathcal{L}(m,-),
    \]
    and 
    \[
        \mathcal{L}_N : N \to {\rm Hom}_R(M,R), \quad n \mapsto \mathcal{L}(-,n),
    \]
    are isomorphisms of $R$-modules.
\end{definition}
Given an $R$-module $M$, we write $M^* := {\rm Hom}_R(M,R)$, and call $M^*$ the 
{\it algebraic dual} of $M$. Consider the $R$-module homomorphisms
\begin{equation}\label{eq:double-dual}
   \sigma_M: M \longrightarrow {\rm Hom}_R(M^*,R), \quad m \mapsto \sigma_M(m) := (f \mapsto f(m)).
\end{equation}
Recall that $M$ is called {\it reflexive} if \eqref{eq:double-dual} is an isomorphism 
(i.e. $\sigma_M$ is bijective). 
\begin{lemma}
    \label{lem:reflexive-perf-pairing}
    \leanok
    Let $M$ be an arbitrary $R$-module, and consider the $R$-bilinear form 
    $\mathcal{L} : M \times M^* \to R, \quad (m,f) \mapsto f(m)$. If $M$ is reflexive, then the
    quadruple $(R,M, M^*, \mathcal{L})$ is a perfect pairing.
\end{lemma}
\begin{proof}
Suppose that $M$ is reflexive. We claim that $M^*$ is also reflexive. To show this, suppose $\sigma_{M^*}(f) = \sigma_{M^*}(g)$ for some $f,g \in M^*$. Then, for every $m \in (M^*)^*$,  $m(f) = \sigma_{M^*}(f)(m) = \sigma_{M^*}(g)(m) = m(g)$. By reflexivity, there exists a unique $n \in M$ such that $\sigma_M(n) = m$, so $f(n) = \sigma_M(n)(f) = \sigma_M(n)(g) = g(n)$, and so $f(n) = g(n)$ for all $n \in M$, which proves injectivity. Surjectivity is similar. To conclude, notice that $\mathcal{L}_M = \sigma_M$ and $\mathcal{L}_{N} = \sigma_{M^*}$.
\end{proof}

If $R$ is a field and $M$ is a finitely generated free $R$-module (i.e. $M$ is a vector space over $R$) then $M \cong M^*$, i.e. there exists $f : M \to M^*$. Then $(f^*)^{-1} \circ f  = \sigma_M$, and $M$ is reflexive. This provides many examples of reflexive modules. 

TODO: Discuss finitely generated projective modules, etc. 


We now turn to the central definition in this section.
\begin{definition}
    \label{def:root-pairing}
    \uses{def:perfect-pairing}
    \leanok
    A {\it root pairing} is a tuple $(R,M,N,\mathcal{L},I,\alpha, \alpha^\vee,s)$ where $(R,M,N,\mathcal{L})$ is a perfect pairing, $I$ is a set (called the {\it index set}),
    \begin{equation}
    \alpha : I \hookrightarrow M, i \mapsto \alpha_i, \quad  \alpha^\vee : I \hookrightarrow N, i \mapsto \alpha_i^\vee \quad \text{and}\quad s : I \to {\rm Perm}(I), i \mapsto s_i
    \end{equation}
    where ${\rm Perm}(I)$ denotes the set of permutations of $I$, subject to the following conditions:\\
       (I) For all $i \in I$, $\mathcal{L}(\alpha_i,\alpha^\vee_i) = 2$. \\
       (II) For all $i,j \in I$, we have
            \begin{align*}
            \alpha_{s_i(j)} &= \alpha_j - \mathcal{L}(\alpha_j,\alpha^\vee_i)\alpha_i \\
            \alpha_{s_i(j)}^\vee &= \alpha_j^\vee - \mathcal{L}(\alpha_i,\alpha^\vee_j)\alpha_i^\vee.
            \end{align*}
\end{definition}
For $i \in I$, we denote by $-i$ the unique element in $I$ such that $s_i(i) = -i$. Note that $s_i = s_{-i}$, and $s_i^2 = {\rm Id}$. For simplicity of notation, for $i,j \in I$, we write $a_{i,j} = \mathcal{L}(\alpha_i,\alpha_j^\vee)$. 

\begin{example}
    \label{eg:eg:ZmodTwoSquared-perf-pairing}
    Let $R=\mathbb{Z}_2 = \mathbb{Z}/2\mathbb{Z}$, $M=R^2$ and $N=M^*$. Write $ M=\{(0,0), (1,0), (0,1), (1,1)\}$ and $N = \{m^T : m \in M\}$, where $(\cdot )^T$ denotes transpose. The bilinear form $\mathcal{L}$ from Lemma \ref{lem:reflexive-perf-pairing} is the usual scalar product between row and columns vectors, i.e. $\mathcal{L}((x,y),{a\choose b})  = ax+by$.
    The resulting quadruple $(R,M,N,\mathcal{L})$ is a perfect pairing by Lemma \ref{lem:reflexive-perf-pairing}.    
    Let $I=\{1,2,3\}$ be our index set, and consider the injective maps $\alpha:I\hookrightarrow M$ and $\alpha^\vee:I\hookrightarrow N$ such that
    \begin{equation*}
        \begin{array}{ccc}
            \alpha_1 = (1,0), & \alpha_2 = (0,1), & \alpha_3 = (1,1), \\
            \alpha^\vee_1 = {0\choose 1}, & \alpha^\vee_2 = {1\choose 0}, &\alpha^\vee_3 = {1\choose 1}.
        \end{array}
    \end{equation*}
    Set $s_1 = (2 3)$, $s_2 = (13)$ and $s_3 = (12)$. Then $(R,M,N,\mathcal{L},I,\alpha,\alpha^\vee,s)$ is a root pairing.
\end{example}


Elements of $\Phi := {\rm Im}(\alpha)$ and $\Phi^\vee := {\rm Im}(\alpha^\vee)$ are called {\it roots} and {\it coroots}, 
respectively, of the root pairing $\mathcal{R}$. If $\mathcal{R} := (R,M,N,\mathcal{L},I,\alpha, \alpha^\vee,s)$ is 
a root pairing then $\mathcal{R}^\vee := (R,N,M,\mathcal{L}^\vee,I,\alpha^\vee, \alpha,s)$ is again 
a root pairing, where $\mathcal{L}^\vee (n,m) := \mathcal{L}(m,n)$ for all $n \in N,m \in M$. The 
latter is called the {\it dual} of $\mathcal{R}$. 

\begin{lemma}
    \label{lem:reflection_perm_unique}
    \leanok
    \uses{def:root-pairing}
    \lean{RootPairing.reflection_perm_unique}
    Given a root pairing $(R,M,N,\mathcal{L},I,\alpha, \alpha^\vee, s)$, the map $s$ is unique. 
    That is, given $t : I \to {\rm Perm}(I)$, the tuple $(R,M,N,\mathcal{L},I,\alpha, \alpha^\vee, t)$
    is a root pairing if and only if $t = s$.
\end{lemma}
\begin{proof}
    \leanok
    Assume to the contrary that $(R,M,N,\mathcal{L},I,\alpha, \alpha^\vee, t)$ is a root pairing, and that 
    there exists $i \in I$ such that $t_i \neq s_i$. In other words, there exists $j \in I$ such that
    $t_i(j) \neq s_i(j)$. By the condition (II),
    \[
        \alpha_{t_i(j)} = \alpha_j - \mathcal{L}(\alpha_j,\alpha^\vee_i)\alpha_i = \alpha_{s_i(j)},
    \]
    which contradicts the assumption that the map $\alpha$ is injective.
\end{proof}


\begin{definition}\label{def:WeylGroup}
\lean{RootPairing.WeylGroup}
\uses{def:root-pairing}
    The {\it Weyl group} of $\mathcal{R}$, denoted $W(\mathcal{R})$, is the subgroup of ${\rm Perm}(I)$ 
    generated by $\{s_i : i \in I\}$. 
\end{definition}

\begin{lemma}\label{lem:pairing-W-invariant}
\lean{RootPairing.pairing_si_invariant,RootPairing.pairing_w_invariant}
\leanok
    \uses{def:root-pairing}
    \uses{def:WeylGroup}
    For any $i,j \in I$, we have 
    \[
        a_{s_i(i),s_i(j)} = a_{i,j} - a_{j,i}a_{i,i}.
    \]
\uses{def:WeylGroup}
    For any $i,j \in I$ and any $w \in W(\mathcal{R})$, we have 
    \[
        a_{w(i),w(j)} = a_{i,j}.
    \]
\end{lemma}
\begin{proof}
    It suffices to prove the claim for $w = s_k$, for $k \in I$. In this case, 
    \begin{equation*}
        a_{s_k(i),s_k(j)} = \mathcal{L}(\alpha_{s_k(i)},\alpha^\vee_{s_k(j)})
        = \mathcal{L}(\alpha_{i} - a_{i,k}\alpha_k,\, \alpha^\vee_{j} - a_{k,j} \alpha_k^\vee)
         = \mathcal{L}(\alpha_{i},\alpha^\vee_{j}).
    \end{equation*}
\end{proof}

\begin{lemma}\label{lem:si_are_reflections}
    \lean{RootPairing.si_are_reflections1,RootPairing.si_are_reflections2}
    \uses{lem:pairing-W-invariant}
    Let $i,j,k \in I$ such that $s_i(j) = k$. Then $s_k = s_is_js_i$. In particular, for 
    $w \in W(\mathcal{R})$ and $w(i) = j$, we have $s_j = ws_iw^{-1}$. 
\end{lemma}
\begin{proof}
To show the first claim, it is enough to prove that for all $m \in I$, $s_is_j(m) = s_ks_i(m)$. Since the map $\alpha$ is injective, it suffices to show that  $\alpha_{s_is_j(m)} = \alpha_{s_ks_i(m)}$. The left-hand side is given by
\begin{equation*}
    \alpha_{s_is_j(m)} = \alpha_{s_j(m)} - a_{s_j(m),i}\alpha_i = \alpha_m - a_{m,j}\alpha_j - a_{s_j(m),i}\alpha_i.
\end{equation*}
The right-hand side is given by
\begin{equation*}
    \alpha_{s_ks_i(m)} = \alpha_{s_i(m)} - a_{s_i(m),k}\alpha_k = \alpha_m - a_{m,i}\alpha_i - a_{s_i(m),k}\alpha_k.
\end{equation*}
But $s_i(j) = k$ implies $s_i(k) = j$ and  $\alpha_k = \alpha_j - a_{j,i}\alpha_i$. Then $a_{s_i(m),k} = a_{m,j}$ by Lemma \ref{lem:pairing-W-invariant}, and therefore
\begin{equation*}
    \alpha_{s_ks_i(m)} = \alpha_m - a_{m,i}\alpha_i - a_{m,j}(\alpha_j - a_{j,i}\alpha_i) = \alpha_m  - a_{m,j}\alpha_j - (a_{m,i} - a_{m,j}a_{j,i})\alpha_i.
\end{equation*}
One checks directly that $a_{s_j(m),i} = a_{m,i}-a_{m,j}a_{j,i}$, which concludes the proof. 
\end{proof}
