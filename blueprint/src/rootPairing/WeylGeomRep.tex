
\section{Geometric representations of the Weyl group}
Fix $\mathcal{R} = (R,M,N,\mathcal{L},I,\alpha, \alpha^\vee,s)$. Consider $\sigma: I \to {\rm End}_R(M)$ and 
$\sigma^\vee : I \to {\rm End}_R(N)$ defined by
\[
    \sigma_i(m) = m - \mathcal{L}(m,\alpha^\vee_i)\alpha_i, \quad \sigma^\vee_i(n) = n - \mathcal{L}(\alpha_i,n)\alpha^\vee_i, \quad m \in M, n \in N.
\]
Then for each $i \in I$, $\sigma_i^2 = {\rm Id}_M$ and $\sigma_i^\vee = {\rm Id}_N$. By Condition (II), 
\[
    \sigma_i(\alpha_j) = \alpha_{s_i(j)}, \quad \sigma^\vee_i(\alpha^\vee_j) = \alpha^\vee_{s_i(j)},
\]
and in particular that $\sigma_i(\Phi) \subset \Phi$ and 
$\sigma_i^\vee(\Phi^\vee) \subset \Phi^\vee$ for all $i \in I$. We denote by $W(\mathcal{R},M)$ and $W(\mathcal{R},N)$ the groups of endomorphisms generated by $\{\sigma_i : i \in I\}$ and $\{\sigma_i^\vee : i \in I\}$ respectively. We refer to either one of these groups as the {\it geometric representation} of $W(\mathcal{R})$.

\begin{lemma}
    For $i,j \in I$, the following are equivalent:
    \begin{enumerate}
        \item $s_i = s_j$,
        \item $2\sigma_i = 2\sigma_j$,
        \item $2\sigma_i^\vee = 2\sigma_j^\vee$.
    \end{enumerate}
    If $2 \in R$ is regular (i.e. the map $r \mapsto 2r$ is injective), $s_i = s_j$ iff
    $\sigma_i = \sigma_j$ iff $\sigma_i^\vee = \sigma_j^\vee$.  
\end{lemma}
\begin{proof}
Suppose that $s_i = s_j$. By condition (II), for all $k \in I$ we have
\[
    a_{k,i}\alpha_i = a_{k,j}\alpha_j \quad \text{and} \quad a_{i,k}\alpha_i^\vee = a_{j,k}\alpha_j^\vee
\]
and in particular, $2\alpha_i = a_{i,j}\alpha_j$ and $a_{i,j}\alpha_i^\vee = 2 \alpha_j^\vee$. Then for all $m \in M$,
\begin{align*}
    2\sigma_i(m) &= 2m - \mathcal{L}(2m,\alpha^\vee_i)\alpha_i \\
                &= 2m - \mathcal{L}(m,\alpha^\vee_i)\, a_{i,j} \,\alpha_j\\
                &= 2m - \mathcal{L}(m,a_{i,j}\, \alpha^\vee_i)\alpha_j\\
                &= 2m - \mathcal{L}(m,2\alpha^\vee_j)\alpha_j\\
                &= 2 \sigma_j(m).
\end{align*}
This shows the equivalence of 1. and 2. The equivalence of 1. and 3. is similar. If $2 \in R$ is regular, 
the statement follows from the equivalence of 1. and 2. and the injectivity of the map $r \mapsto 2r$.
\end{proof}

\textsc{Comment about injectivity of the representation}






Remaining concepts to be added:
\begin{enumerate}
    \item Values of the root pairing and crystallographic root pairings
    \item Coxeter weights
    \item Orthogonal roots 
    \item Irreducible root pairings
    \item Reduced root pairings
\end{enumerate}