\chapter{Root pairing}
\label{cha:root-pairing}
\section{Basic definitions}
\label{sec:basic-definitions}

We follow the outline of mathlib's definitions. 

\begin{definition}
    A {\it perfect pairing} is a quadruple $(R,M,N,\mathcal{L})$ where $R$ is a commutative ring, $M,N$ are 
    $R$-modules, and $\mathcal{L} : M \times N \to R$ is an $R$-bilinear map such that the induced maps
    \[
        \mathcal{L}_M : M \to {\rm Hom}_R(N,R), \quad m \mapsto \mathcal{L}(m,-),
    \]
    and 
    \[
        \mathcal{L}_N : N \to {\rm Hom}_R(M,R), \quad n \mapsto \mathcal{L}(-,n),
    \]
    are isomorphisms of $R$-modules.
\end{definition}

This leads to the very general notion of a root pairing.
\begin{definition}
    A {\it root pairing} is a tuple $(R,M,N,\mathcal{L},I,\alpha, \alpha^\vee,s)$ with
       \begin{enumerate} 
        \item $(R,M,N,\mathcal{L})$ is a perfect pairing,
        \item  $I$ is a set (called the {\it index set}),
        \item $\alpha : I \to M, i \mapsto \alpha_i$ is an injective map on sets,
        \item $\alpha^\vee : I \to N, i \mapsto \alpha_i^\vee$ is an injective map on sets,
        \item $s : I \to {\rm Bij}(I,I)$, i.e. for each $i \in I$, $s_i \stackrel{\text{denotes}}{=}s(i)$ is a bijection on $I$,
       \end{enumerate}
       subject to the following conditions:
       \begin{enumerate}[label=\Roman*]
            \item For all $i \in I$, $\mathcal{L}(\alpha_i,\alpha^\vee_i) = 2$.
            \item For all $i,j \in I$, we have
            \begin{align*}
            \alpha_{s_i(j)} &= \alpha_j - \mathcal{L}(\alpha_j,\alpha^\vee_i)\alpha_i \\
            \alpha_{s_i(j)}^\vee &= alpha_j^\vee - \mathcal{L}(\alpha_i,\alpha^\vee_j)\alpha_i^\vee
            \end{align*}
       \end{enumerate}
\end{definition}

{\it Lean implementation note: A root pairing is defined in Mathlib as a {\it structure}. A structure is a non-recursive inductive 
type that contains only one constructor. It is also sometimes called a {\it record}. Formally, one can construct a structure using either
the ''inductive'' (with only one constructor, and including field names) or the ''structure'' keyword. The latter is more convenient for defining a structure,
since Lean automatically generates the projection functions for each field. If the constructor name is not provided when defining a structure, then a 
constructor is named ``mk" by default.}



