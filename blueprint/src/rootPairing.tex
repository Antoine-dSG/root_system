\chapter{Root pairing}
\label{cha:root-pairing}
\section{Basic definitions}
\label{sec:basic-definitions}

We follow the outline of mathlib's definitions. 

\begin{definition}
   \label{def:perfect-pairing}
   \leanok
    A {\it perfect pairing} is a quadruple $(R,M,N,\mathcal{L})$ where $R$ is a commutative ring, $M,N$ are 
    $R$-modules, and $\mathcal{L} : M \times N \to R$ is an $R$-bilinear map such that the induced maps
    \[
        \mathcal{L}_M : M \to {\rm Hom}_R(N,R), \quad m \mapsto \mathcal{L}(m,-),
    \]
    and 
    \[
        \mathcal{L}_N : N \to {\rm Hom}_R(M,R), \quad n \mapsto \mathcal{L}(-,n),
    \]
    are isomorphisms of $R$-modules.
\end{definition}
Given an $R$-module $M$, we write $M^* := {\rm Hom}_R(M,R)$, and call $M^*$ the 
{\it algebraic dual} of $M$. Consider the $R$-module homomorphisms
\begin{equation}\label{eq:double-dual}
   \sigma_M: M \longrightarrow {\rm Hom}_R(M^*,R), \quad m \mapsto \sigma_M(m) := (f \mapsto f(m)).
\end{equation}
Recall that $M$ is called {\it reflexive} if \eqref{eq:double-dual} is an isomorphism 
(i.e. $\sigma_M$ is bijective). 
\begin{lemma}\label{lem:reflexive-perf-pairing}
    Let $M$ be an arbitrary $R$-module, and consider the $R$-bilinear form 
    $\mathcal{L} : M \times M^* \to R, \quad (m,f) \mapsto f(m)$. If $M$ is reflexive, then the
    quadruple $(R,M, M^*, \mathcal{L})$ is a perfect pairing.
\end{lemma}
\begin{proof}
Suppose that $M$ is reflexive. We claim that $M^*$ is also reflexive. To show this, suppose $\sigma_{M^*}(f) = \sigma_{M^*}(g)$ for some $f,g \in M^*$. Then, for every $m \in (M^*)^*$,  $m(f) = \sigma_{M^*}(f)(m) = \sigma_{M^*}(g)(m) = m(g)$. By reflexivity, there exists a unique $n \in M$ such that $\sigma_M(n) = m$, so $f(n) = \sigma_M(n)(f) = \sigma_M(n)(g) = g(n)$, and so $f(n) = g(n)$ for all $n \in M$, which proves injectivity. Surjectivity is similar. To conclude, notice that $\mathcal{L}_M = \sigma_M$ and $\mathcal{L}_{N} = \sigma_{M^*}$.
\end{proof}

If $R$ is a field and $M$ is a finitely generated free $R$-module (i.e. $M$ is a vector space over $R$) then $M \cong M^*$, i.e. there exists $f : M \to M^*$. Then $(f^*)^{-1} \circ f  = \sigma_M$, and $M$ is reflexive. This provides many examples of reflexive modules. 

TODO: Discuss finitely generated projective modules, etc. 
\begin{example}
    \label{eg:ZmodTwoSquared-reflexive}
    Let $R=\mathbb{Z}_2 = \mathbb{Z}/2\mathbb{Z}$, $M=R^2$ and $N=M^*$. Explicitly, we can write $ M=\{(0,0), (1,0), (0,1), (1,1)\}$. As a $\mathbb{Z}_2$-module, $M$ is generated by the subset $\{(1,0), (0,1)\}$, so elements in $N$ are determined by where these elements are mapped to in $\mathbb{Z}_2$.
    There are hence four elements of $N$, namely
    \begin{align*}
        f_{0\choose0}&: (1,0)\mapsto 0,\quad (0,1)\mapsto 0,\\
        f_{1\choose0}&: (1,0)\mapsto 1,\quad (0,1)\mapsto 0,\\
        f_{0\choose1}&: (1,0)\mapsto 0,\quad (0,1)\mapsto 1,\\
        f_{1\choose1}&: (1,0)\mapsto 1,\quad (0,1)\mapsto 1.
    \end{align*}
    Going forward we'll omit the ``$f$''.
    We now have a $\mathbb{Z}_2$ bilinear map $\langle{-},{-}\rangle:M\times N \to \mathbb{Z}_2$ given by evaluation, $\langle (x,y),{a\choose b} \rangle = ax+by$.
    This structure of $N=M^*$ shows that $M\cong M^*$ as $\mathbb{Z}_2$-modules, namely via the isomorphism $(x,y) \mapsto {x\choose y}$. This implies that $M\cong N \cong N^* = (M^*)^*$, so $M$ is a reflexive module and we have a perfect pairing by Lemma \ref{lem:reflexive-perf-pairing}    
\end{example}



We now turn to the central definition in this section.
\begin{definition}
    \label{def:root-pairing}
   \uses{def:perfect-pairing}
   \leanok
    A {\it root pairing} is a tuple $(R,M,N,\mathcal{L},I,\alpha, \alpha^\vee,s)$ where
       \begin{enumerate} 
        \item $(R,M,N,\mathcal{L})$ is a perfect pairing,
        \item  $I$ is a set (called the {\it index set}),
        \item $\alpha : I \to M, i \mapsto \alpha_i$ is an injective map on sets,
        \item $\alpha^\vee : I \to N, i \mapsto \alpha_i^\vee$ is an injective map on sets,
        \item $s : I \to {\rm Perm}(I), i \mapsto s_i$ (here, ${\rm Perm}(I)$ denotes the set of permutations of $I$),
       \end{enumerate}
       subject to the following conditions:\\
       I) For all $i \in I$, $\mathcal{L}(\alpha_i,\alpha^\vee_i) = 2$. \\
       II) For all $i,j \in I$, we have
            \begin{align*}
            \alpha_{s_i(j)} &= \alpha_j - \mathcal{L}(\alpha_j,\alpha^\vee_i)\alpha_i \\
            \alpha_{s_i(j)}^\vee &= \alpha_j^\vee - \mathcal{L}(\alpha_i,\alpha^\vee_j)\alpha_i^\vee
            \end{align*}
\end{definition}
Note that the injectivity of $\alpha$ guarantees $s_i^2 = {\rm Id}$ for all $i \in I$.

\begin{example}
    \label{eg:eg:ZmodTwoSquared-perf-pairing}
    We continue from Example \ref{eg:ZmodTwoSquared-reflexive}, so $R=\mathbb{Z}_2 = \mathbb{Z}/2\mathbb{Z}$, $M=(\mathbb{Z}_2)^2$, $N=M^*=\mathrm{Hom}_{R}(M,R)$, with the perfect pairing given by evaluation, and extend this to a root pairing.
    Let $I=\{1,2,3\}$ be our index set, and consider the injective maps $\alpha:I\to M$ and $\alpha^\vee:I\to N$ such that
    \begin{equation*}
        \begin{array}{ccc}
            \alpha_1 = (1,0), & \alpha_2 = (0,1), & \alpha_3 = (1,1), \\
            \alpha^\vee_1 = {0\choose 1}, & \alpha^\vee_2 = {1\choose 0}, &\alpha^\vee_3 = {1\choose 1}.
        \end{array}
    \end{equation*}
    We now see that condition (I) is satisfied, as
    \begin{equation*}
        \langle\alpha_i,\alpha_i^\vee\rangle = \begin{cases}\langle (1,0),{0\choose 1} \rangle =0 = 2 &\text{ when } i=1,\\\langle (0,1),{1\choose 0} \rangle =0 = 2 &\text{ when } i=2,\\\langle (1,1),{1\choose 1} \rangle = 2 &\text{ when } i=3.\\
        \end{cases}
    \end{equation*}
    For each $i\in I$, we need a permutation $s_i$. Define,
    $$s_1=\begin{pmatrix}
        1&2 &3 \\ 1 &3 &2
    \end{pmatrix},\quad
    s_2=\begin{pmatrix}
        1&2 &3 \\ 3 &2 &1
    \end{pmatrix},\quad
    s_3=\begin{pmatrix}
        1&2 &3 \\ 2 &1 &3
    \end{pmatrix}.$$
    To verify condition (II), first observe that both equations are satisfied when $i=j$ as we have shown that $\langle\alpha_i,\alpha_i^\vee\rangle=2=0$ in checking condition (I).
    When $i=1$, we have for $j=2$,
    \begin{align*}
        \alpha_{s_1(2)} = \alpha_3 = \alpha_2 - 1\cdot\alpha_1=  \alpha_2 - \langle\alpha_2,\alpha_1^\vee\rangle\alpha_1 \text{ and }
            \alpha_{s_1(2)}^\vee = \alpha_3^\vee = \alpha_2^\vee -1\cdot \alpha_1^\vee= \alpha_2^\vee - \langle\alpha_1,\alpha^\vee_2\rangle\alpha_1^\vee
    \end{align*}
    and when $j=3$, we have
    \begin{align*}
        \alpha_{s_1(3)} = \alpha_2 = \alpha_3 - 1\cdot\alpha_1=  \alpha_3 - \langle\alpha_3,\alpha_1^\vee\rangle\alpha_1 \text{ and }
            \alpha_{s_1(3)}^\vee = \alpha_2^\vee = \alpha_3^\vee -1\cdot \alpha_1^\vee= \alpha_3^\vee - \langle\alpha_1,\alpha^\vee_3\rangle\alpha_1^\vee.
    \end{align*}
    
    The other cases are similar using the symmetry of these roots and coroots as generators of $M$ and $N$.
\end{example}


Elements of $\Phi := {\rm Im}(\alpha)$ and $\Phi^\vee := {\rm Im}(\alpha^\vee)$ are called {\it roots} and {\it coroots}, 
respectively, of the root pairing $\mathcal{R}$. If $\mathcal{R} := (R,M,N,\mathcal{L},I,\alpha, \alpha^\vee,s)$ is 
a root pairing then $\mathcal{R}^\vee := (R,N,M,\mathcal{L}^\vee,I,\alpha^\vee, \alpha,s)$ is again 
a root pairing, where $\mathcal{L}^\vee (n,m) := \mathcal{L}(m,n)$ for all $n \in N,m \in M$. The 
latter is called the {\it dual} of $\mathcal{R}$. 

\begin{lemma}
    Given a root pairing $(R,M,N,\mathcal{L},I,\alpha, \alpha^\vee, s)$, the map $s$ is unique. 
    That is, given $t : I \to {\rm Perm}(I)$, the tuple $(R,M,N,\mathcal{L},I,\alpha, \alpha^\vee, t)$
    is a root pairing if and only if $t = s$.
\end{lemma}
\begin{proof}
    Assume to the contrary that $(R,M,N,\mathcal{L},I,\alpha, \alpha^\vee, t)$ is a root pairing, and that 
    there exists $i \in I$ such that $t_i \neq s_i$. In other words, there exists $j \in I$ such that
    $t_i(j) \neq s_i(j)$. By the condition (II),
    \[
        \alpha_{t_i(j)} = \alpha_j - \mathcal{L}(\alpha_j,\alpha^\vee_i)\alpha_i = \alpha_{s_i(j)},
    \]
    which contradicts the assumption that the map $\alpha$ is injective.
\end{proof}


\begin{definition}
    The {\it Weyl group} of $\mathcal{R}$, denoted $W(\mathcal{R})$, is the subgroup of ${\rm Perm}(I)$ 
    generated by $\{s_i : i \in I\}$. 
\end{definition}
\begin{lemma}
    Let $i,j,k \in I$ such that $s_i(j) = k$. Then $s_k = s_is_js_i$. In particular, for 
    $w \in W(\mathcal{R})$ and $w(i) = j$, we have $s_j = ws_iw^{-1}$. 
\end{lemma}
\begin{proof}

\end{proof}

\begin{lemma}
\lean{RootPairing.pairing_si_invariant}
    For any $i,j \in I$ and any $w \in W(\mathcal{R})$, we have 
    \[
        \mathcal{L}(\alpha_{w(i)},\alpha^\vee_{w(j)}) = \mathcal{L}(\alpha_{i},\alpha_j^\vee).
    \]
\end{lemma}
\begin{proof}
    It suffices to prove the claim for $w = s_k$, for some $k \in I$. In this case, 
    \begin{align*}
        \mathcal{L}(\alpha_{s_k(i)},\alpha^\vee_{s_k(j)}) &=  \mathcal{L}(\sigma_k(\alpha_{i}),\sigma_k^\vee(\alpha^\vee_{j}))\\
                                &= \mathcal{L}(\alpha_{i} - \mathcal{L}(\alpha_i,\alpha_k^\vee)\alpha_k,\alpha^\vee_{j} - \mathcal{L}(\alpha_k,\alpha_j^\vee) \alpha_k^\vee)\\
                            & = \mathcal{L}(\alpha_{i},\alpha^\vee_{j}).
    \end{align*}
\end{proof}

\section{Geometric representations of the Weyl group}
Let us fix a root pairing $\mathcal{R} = (R,M,N,\mathcal{L},I,\alpha, \alpha^\vee,s)$. Consider $\sigma: I \to {\rm End}_R(M)$ and 
$\sigma^\vee : I \to {\rm End}_R(N)$ defined by
\[
    \sigma_i(m) = m - \mathcal{L}(m,\alpha^\vee_i)\alpha_i, \quad \sigma^\vee_i(n) = n - \mathcal{L}(\alpha_i,n)\alpha^\vee_i, \quad m \in M, n \in N.
\]
Then $\sigma_i$ and $\sigma^\vee_i$ are $R$-module homomorphisms. Moreover, 
for each $i \in I$, $\sigma_i^2 = {\rm Id}_M$ and $\sigma_i^\vee = {\rm Id}_N$. By Condition (II), 
\[
    \sigma_i(\alpha_j) = \alpha_{s_i(j)}, \quad \sigma^\vee_i(\alpha^\vee_j) = \alpha^\vee_{s_i(j)},
\]
and in particular that $\sigma_i(\Phi) \subset \Phi$ and 
$\sigma_i^\vee(\Phi^\vee) \subset \Phi^\vee$ for all $i \in I$.

\begin{lemma}
    For $i,j \in I$, the following are equivalent:
    \begin{enumerate}
        \item $s_i = s_j$,
        \item $2\sigma_i = 2\sigma_j$,
        \item $2\sigma_i^\vee = 2\sigma_j^\vee$.
    \end{enumerate}
    If $2 \in R$ is regular (i.e. the map $r \mapsto 2r$ is injective), $s_i = s_j$ iff
    $\sigma_i = \sigma_j$ iff $\sigma_i^\vee = \sigma_j^\vee$.  
\end{lemma}
\begin{proof}
Suppose that $s_i = s_j$. By condition (II), for all $k \in I$ we have
\[
    \mathcal{L}(\alpha_k,\alpha_i^\vee)\alpha_i = \mathcal{L}(\alpha_k,\alpha_j^\vee)\alpha_j,
\]
and in particular, $2\alpha_i = \mathcal{L}(\alpha_i,\alpha_j^\vee)\alpha_j$. Then for all $m \in M$,
\begin{align*}
    2\sigma_i(m) &= 2(m - \mathcal{L}(m,\alpha^\vee_i)\alpha_i) \\
                &= 2m - \mathcal{L}(m,\alpha^\vee_i)\mathcal{L}(\alpha_i,\alpha_j^\vee)\alpha_j\\
                &= 2m - \mathcal{L}(m,\mathcal{L}(\alpha_i,\alpha_j^\vee)\alpha^\vee_i)\alpha_j\\
                 &= 2m - \mathcal{L}(m,\alpha^\vee_j)2\alpha_j\\
                &= 2 \sigma_j(m).
\end{align*}
This shows the equivalence of 1. and 2. The equivalence of 1. and 3. is similar. If $2 \in R$ is regular, 
the statement follows from the equivalence of 1. and 2. and the injectivity of the map $r \mapsto 2r$.
\end{proof}

\textsc{Comment about injectivity of the representation}






Remaining concepts to be added:
\begin{enumerate}
    \item Values of the root pairing and crystallographic root pairings
    \item Coxeter weights
    \item Orthogonal roots 
    \item Irreducible root pairings
    \item Reduced root pairings
\end{enumerate}