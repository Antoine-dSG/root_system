\chapter{Root systems}
\label{cha:root-systems}
\section{Finite root systems}
\label{sec:finite-root-systems}

Throughout, we fix $k$ to be a field. 

\begin{definition}
    Let $V$ be a vector space over $k$. A linear map $s : V \to V$ is called a {\it pseudo-reflection} if 
    the rank of ${\rm Id}-s$ is $1$. 
\end{definition}
Denote by $D$ the image of ${\rm Id}-s$. By definition, $D$ is a one-dimensional vector subspace of $V$. 
Thus, given $a \in D$ with $a \neq 0$, there exists some $a^* \in V^*$ such that $({\rm Id}-s)(x)= a^*(x) a$ for 
all $x \in V$. Moreover, $ker(a^*) = D$.

\begin{definition}
    Let $s$ be a pseudo-reflection in $V$. Then, $s$ is called a {\it reflection} if $s^2 = {\rm Id}$.
\end{definition}

Note that if $s$ a reflection in $V$, then 
    \[
        V = ker({\rm Id}-s) \oplus ker(s+{\rm Id}).
    \]

\begin{definition}
  \label{def:root-system}
  Let $V$ be a vector space over $k$, $\Phi$ a subset of $V$. We say that $\Phi$ is a 
  \emph{(finite) root system} in $V$ if the following conditions are satisfied:
    \begin{enumerate}
        \item $\Phi$ is finite,
        \item $\Phi$ spans $V$.
        \item For every $\alpha$ in $\Phi$, there exists an element $\alpha^\vee \in V^*$ such that 
            $\alpha^\vee(\alpha) = 2$, and the reflection $s_{\alpha,\alpha^\vee}: V \to V$ fixes $\Phi$.
        \item (Crystallographic condition) For every $\alpha\in \Phi$, we have $\alpha^\vee(\Phi) \subset \mathbb{Z}$
    \end{enumerate}
\end{definition}
Given a root system $\Phi \subset V$, Definition \ref{def:root-system} implies the existence of a 
{\it a priori non-unique} set-theoretic injective map $\Phi \to V^*$ given by $\alpha \mapsto \alpha^\vee$. 
We denote by $\Phi^\vee$ the image of this map, and call its elements {\it coroots}. 

\textsc{When are the coroots uniquely determined by the pair $(V,\Phi)$?}



